% A book summary template inspired by Jan Küster's Left Sidebar CV / https://github.com/jankapunkt/latexcv

\documentclass[11pt,a4paper]{ctexart}	
\usepackage[utf8]{inputenc}	

% some tex-live fonts - choose your own

%\usepackage[defaultsans]{droidsans}
%\usepackage[default]{comfortaa}
%\usepackage{cmbright}
\usepackage[default]{raleway}
%\usepackage{fetamont}
%\usepackage[default]{gillius}
%\usepackage[light,math]{iwona}
%\usepackage[thin]{roboto}

% set font default
\renewcommand*\familydefault{\sfdefault} 	
\usepackage[T1]{fontenc}

\usepackage{moresize}
\usepackage{fontawesome}

\usepackage{paracol}
\usepackage[margin=1.5cm]{geometry}

\usepackage{fancyhdr}
\pagestyle{empty}
\setlength{\parindent}{0mm}
\usepackage{graphicx}
	
\usepackage{tikz}				
\usetikzlibrary{shapes, backgrounds,mindmap, trees}

\usepackage{transparent}
\usepackage{color}

\definecolor{maincol}{RGB}{ 225, 0, 0 }
\definecolor{darkcol}{RGB}{ 70, 70, 70 }
\definecolor{lightcol}{RGB}{245,245,245}

\usepackage{enumitem}
\setitemize{label={\color{maincol}\faCheck}}

\usepackage[hidelinks]{hyperref}
\include{kuestercvelements}


%============================================================================%
\begin{document}

\columnratio{0.618}
\setlength{\columnsep}{2.2em}
\setlength{\columnseprule}{0.618pt}
\colseprulecolor{darkcol}
\begin{paracol}{2}

%\includegraphics[width=\linewidth]{bookcover.jpg}


%\begin{figure}[h]
%    \centering
%    \includegraphics[width=\linewidth]{figure1.jpg}
    %\caption{Graph illustrating how far you can go with the same amount of energy if invested deliberately in just one thing versus diffusion}
%    \label{fig:my_label}
%\end{figure}





\normalsize


% \titlebox{white}{darkcol}{
% \bigfont{数分习题课讲义}

% \titletext{第01章 引论}

% }
% \vspace{1em}

%---------------------------------------------------------------------------------------
\heading{2.1.2思考题}

\includegraphics[width=\linewidth]{figure01.png}
\newpage
\heading{2.1.5练习题}

\includegraphics[width=\linewidth]{figure02.png}
\newpage
\heading{2.2.1思考题}

\includegraphics[width=\linewidth]{figure03.png}
\newpage
\heading{2.2.4练习题}

\includegraphics[width=\linewidth]{figure04.png}
\newpage
\heading{2.3.2练习题}

\includegraphics[width=\linewidth]{figure05.png}
\newpage
\heading{2.4.3练习题}

\includegraphics[width=\linewidth]{figure06.png}
\newpage
\heading{2.5.5练习题}

\includegraphics[width=\linewidth]{figure07.png}
\newpage
\heading{2.6.3练习题}

\includegraphics[width=\linewidth]{figure08.png}
\newpage
\heading{2.7.3第1组参考题}

\includegraphics[width=\linewidth]{figure09.png}
\newpage
\heading{2.7.3第1组参考题}

\includegraphics[width=\linewidth]{figure10.png}
\newpage
\heading{2.7.3第2组参考题}

\includegraphics[width=\linewidth]{figure11.png}
\newpage
\heading{2.7.3第2组参考题}

\includegraphics[width=\linewidth]{figure12.png}
\newpage
\heading{2.8第1次习题课}

\includegraphics[width=\linewidth]{figure13.png}
\newpage
\heading{2.8第2次习题课}

\includegraphics[width=\linewidth]{figure14.png}
\newpage
\heading{2.8第3次习题课}

\includegraphics[width=\linewidth]{figure15.png}
\newpage
\heading{2.8第4次习题课}

\includegraphics[width=\linewidth]{figure16.png}
\newpage

% \begin{figure}[h]
%    \centering
%    \includegraphics[width=\linewidth]{figure01.png}
% \end{figure}

% \begin{figure}[h]
%    \centering
%    \includegraphics[width=\linewidth]{figure02.png}
% \end{figure}

\small
\switchcolumn

\heading{答题区}
\newpage
\heading{答题区}
\newpage
\heading{答题区}
\newpage
\heading{答题区}
\newpage
\heading{答题区}
\newpage
\heading{答题区}
\newpage
\heading{答题区}
\newpage
\heading{答题区}
\newpage
\heading{答题区}
\newpage
\heading{答题区}
\newpage
\heading{答题区}
\newpage
\heading{答题区}
\newpage
\heading{答题区}
\newpage
\heading{答题区}
\newpage
\heading{答题区}
\newpage
\heading{答题区}
\newpage
\end{paracol}





\end{document}

